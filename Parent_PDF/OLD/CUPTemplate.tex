\documentclass[
  journal=largetwo,
  manuscript=article-type,
  year=2020,
  volume=37,
]{cup-journal}

\usepackage{amsmath}
\usepackage[nopatch]{microtype}
\usepackage{booktabs}

\title{Cambridge Large2 Template Class File}

\author{F. Author}
\affiliation{First Division, Organization, City, Pincode, State, Country}
\email[F. Author]{first.author@address.edu}

\author{S. Author}
\affiliation{Second Division, Organization, City, Pincode, State, Country}
% \alsoaffiliation{Joint first authors}

\author{T. Author}
\affiliation{Second Division, Organization, City, Pincode, State, Country}

\author{F.T. Author}
\affiliation{Fourth Division, Organization, City, Pincode, State, Country}

\addbibresource{example.bib}

\keywords{keyword entry 1, keyword entry 2, keyword entry 3} %% First letter not capped

\begin{document}

\begin{abstract}
Insert abstract text here. Lorem ipsum dolor sit amet, consectetur adipiscing elit, sed do eiusmod tempor incididunt ut labore et dolore magna aliqua. Lorem ipsum dolor sit amet, consectetur adipiscing elit, sed do eiusmod tempor incididunt ut labore et dolore magna aliqua. Lorem ipsum dolor sit amet, consectetur adipiscing elit, sed do eiusmod tempor incididunt ut labore et dolore magna aliqua. 
\end{abstract}

\noindent Lorem ipsum dolor sit amet, consectetur adipiscing elit, sed do eiusmod tempor incididunt ut labore et dolore magna aliqua. 

\section{Insert A head here}
This demo file is intended to serve as a ``starter file''. It is for preparing manuscript submission only, not for preparing camera-ready versions of manuscripts. Manuscripts will be typeset for publication by the journal, after they have been accepted.

By default, this template uses \texttt{biblatex} and adopts the Chicago referencing style. If you are using this template on Overleaf, Overleaf's build tool will automatically run \texttt{pdflatex} and \texttt{biber}. If you are compiling this template on your own local \LaTeX{} installation, please execute the following commands:
\begin{enumerate}
    \item \verb|pdflatex sample|
    \item \verb|biber sample|
    \item \verb|pdflatex sample|
    \item \verb|pdflatex sample|
\end{enumerate}

Some journals e.g.~\texttt{journal=aog|jog|pasa}
\\require Bib\TeX{}. For such journals, you will need to
\begin{itemize}
    \item delete the existing \verb|\addbibresource{example.bib}|;
    \item change the existing \verb|\printbibliography| to be\\
    \verb|\bibliography{example}| instead.
\end{itemize} 

Overleaf will run \texttt{pdflatex} and \texttt{bibtex} automatically as needed. But if you had \emph{first} compiled using another \texttt{journal} option that adopts \texttt{biblatex}, and \emph{then} change the \texttt{journal} option to one that adopts Bib\TeX{}, you may get some compile error messages instead. In this case you will need to do a `Recompile from scratch'; see \url{https://www.overleaf.com/learn/how-to/Clearing_the_cache}.

On a local \LaTeX{} installation, you would need to run these steps instead:
  \begin{enumerate}
\item Delete \texttt{sample.aux}, \texttt{sample.bbl} if these files from a previous compile using \texttt{biber} still exist.
\item \verb|pdflatex sample|
  \item \verb|bibtex sample|
  \item \verb|pdflatex sample|
  \item \verb|pdflatex sample|
  \end{enumerate}

Lorem ipsum dolor sit amet, consectetur adipiscing elit, sed do eiusmod tempor incididunt ut labore et dolore magna aliqua. 

Lorem ipsum dolor sit amet, consectetur adipiscing elit, sed do eiusmod tempor incididunt ut labore et dolore magna aliqua. Lorem ipsum dolor sit amet, consectetur adipiscing elit, sed do eiusmod tempor incididunt ut labore et dolore magna aliqua. 


\subsection{Insert B head here}
Subsection text here. Lorem ipsum\autocite{Bayer_etal_2013} dolor sit amet, consectetur adipiscing elit, sed do eiusmod tempor incididunt ut labore\autocite{Adade_etal_2007} et dolore magna aliqua. 

Lorem ipsum dolor sit amet, consectetur adipiscing elit, sed do eiusmod tempor incididunt ut labore et dolore magna aliqua. Lorem ipsum dolor sit amet, consectetur adipiscing elit, sed do eiusmod tempor incididunt ut labore et dolore magna aliqua. Lorem ipsum dolor sit amet, consectetur adipiscing elit, sed do eiusmod tempor incididunt ut labore et dolore magna aliqua. 

\subsubsection{Insert C head here}
Subsubsection text here. Lorem ipsum dolor sit amet, consectetur adipiscing elit, sed do eiusmod tempor incididunt ut labore et dolore magna aliqua. 
Lorem ipsum dolor sit amet, consectetur adipiscing elit, sed do eiusmod tempor incididunt ut labore et dolore magna aliqua. 

Lorem ipsum dolor sit amet, consectetur adipiscing elit, sed do eiusmod tempor incididunt ut labore et dolore magna aliqua. Lorem ipsum dolor sit amet, consectetur adipiscing elit, sed do\endnote{A footnote/endnote} eiusmod tempor incididunt ut labore et dolore magna aliqua. 

\section{Equations}

Sample equations. Lorem ipsum dolor sit amet, consectetur adipiscing elit, sed do eiusmod tempor incididunt ut labore et dolore magna aliqua. Lorem ipsum dolor sit amet, consectetur\endnote{Another footnote/endnote} adipiscing elit, sed do eiusmod tempor incididunt ut labore et dolore magna aliqua. Lorem ipsum dolor sit amet, consectetur adipiscing elit, sed do eiusmod tempor incididunt ut labore et dolore magna aliqua. 


%%% Numbered equation
\begin{equation}
\begin{aligned}\label{eq:first}
\frac{\partial u(t,x)}{\partial t} = Au(t,x) \left(1-\frac{u(t,x)}{K}\right)
-B\frac{u(t-\tau,x) w(t,x)}{1+Eu(t-\tau,x)},\\
\frac{\partial w(t,x)}{\partial t} =\delta \frac{\partial^2w(t,x)}{\partial x^2}-Cw(t,x)
+D\frac{u(t-\tau,x)w(t,x)}{1+Eu(t-\tau,x)},
\end{aligned}
\end{equation}

Lorem ipsum dolor sit amet, consectetur adipiscing elit, sed do eiusmod tempor incididunt ut labore et dolore magna aliqua. Lorem ipsum dolor sit amet, consectetur adipiscing elit, sed do eiusmod tempor incididunt ut labore et dolore magna aliqua. Lorem ipsum dolor sit amet, consectetur adipiscing elit, sed do eiusmod tempor incididunt ut labore et dolore magna aliqua. 

\begin{align}\label{eq:another}
\begin{split}
\frac{dU}{dt} &=\alpha U(t)(\gamma -U(t))-\frac{U(t-\tau)W(t)}{1+U(t-\tau)},\\
\frac{dW}{dt} &=-W(t)+\beta\frac{U(t-\tau)W(t)}{1+U(t-\tau)}.
\end{split}
\end{align}


%%%% Unnumbered equation
\begin{align*}
&\frac{\partial(F_1,F_2)}{\partial(c,\omega)}_{(c_0,\omega_0)} = \left|
  \begin{array}{ll}
\frac{\partial F_1}{\partial c} &\frac{\partial F_1}{\partial \omega} \\\noalign{\vskip3pt}
\frac{\partial F_2}{\partial c}&\frac{\partial F_2}{\partial \omega}
\end{array}\right|_{(c_0,\omega_0)}\\
&\quad=-4c_0q\omega_0 -4c_0\omega_0p^2 =-4c_0\omega_0(q+p^2)>0.
\end{align*}


\section{Figures \& Tables}

The output for a single-column figure is in Figure~\ref{fig_sim}.  Lorem ipsum dolor sit amet, consectetur adipiscing elit, sed do eiusmod tempor incididunt ut labore et dolore magna aliqua. Lorem ipsum dolor sit amet, consectetur adipiscing elit, sed do eiusmod tempor incididunt ut labore et dolore magna aliqua. Lorem ipsum dolor sit amet, consectetur adipiscing elit, sed do eiusmod tempor incididunt ut labore et dolore magna aliqua. 

Lorem ipsum dolor sit amet, consectetur adipiscing elit, sed do eiusmod tempor incididunt ut labore et dolore magna aliqua. Lorem ipsum dolor sit amet, consectetur adipiscing elit, sed do eiusmod tempor incididunt ut labore et dolore magna aliqua. Lorem ipsum dolor sit amet, consectetur adipiscing elit, sed do eiusmod tempor incididunt ut labore et dolore magna aliqua. 

%See Figure~\ref{fig_wide} for a double-column figure; this is always at the top of a following page.


\begin{figure}[hbt!]
\centering
\includegraphics[width=0.75\linewidth]{example-image-16x10.pdf}
\caption{Insert figure caption here}
\label{fig_sim}
\end{figure}


\begin{figure*}
\centering
\includegraphics[width=0.8\linewidth]{example-image-16x10.pdf}
\caption{Insert figure caption here}
\label{fig_wide}
\end{figure*}


See example table in Table~\ref{table_example}.

\begin{table}[hbt!]
\begin{threeparttable}
\caption{An Example of a Table}
\label{table_example}
\begin{tabular}{llll}
\toprule
\headrow Column head 1 & Column head 2  & Column head 3 & Column head 4\\
\midrule
One\tnote{a} & Two&three three &four\\ 
\midrule
Three & Four&three three\tnote{b} &four\\
\bottomrule
\end{tabular}
\begin{tablenotes}[hang]
\item[]Table note
\item[a]First note
\item[b]Another table note
\end{tablenotes}
\end{threeparttable}
\end{table}


\section{Conclusion}
The conclusion text goes here.


\begin{acknowledgement}
Insert the Acknowledgment text here.
\end{acknowledgement}

\paragraph{Funding Statement}

This research was supported by grants from the <funder-name> <doi> (<award ID>); <funder-name> <doi> (<award ID>).

\paragraph{Competing Interests}

A statement about any financial, professional, contractual or personal relationships or situations that could be perceived to impact the presentation of the work --- or `None' if none exist.

\paragraph{Data Availability Statement}

A statement about how to access data, code and other materials allowing users to understand, verify and replicate findings --- e.g. Replication data and code can be found in Harvard Dataverse: \verb+\url{https://doi.org/link}+.



%\endnote in some journals will behave like \footnote; and \printendnotes will not output anything. 
\printendnotes

\printbibliography

\appendix

\section{Example Appendix Section}

Lorem ipsum dolor sit amet, consectetur adipiscing elit, sed do eiusmod tempor incididunt ut labore et dolore magna aliqua. Lorem ipsum dolor sit amet, consectetur adipiscing elit, sed do eiusmod tempor incididunt ut labore et dolore magna aliqua. Lorem ipsum dolor sit amet, consectetur adipiscing elit, sed do eiusmod tempor incididunt ut labore et dolore magna aliqua. 

\end{document}