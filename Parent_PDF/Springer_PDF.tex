% !TeX program = pdfLaTeX
\documentclass[smallextended]{svjour3}       % onecolumn (second format)
%\documentclass[twocolumn]{svjour3}          % twocolumn
%
\smartqed  % flush right qed marks, e.g. at end of proof
%
\usepackage{amsmath}
\usepackage{graphicx}
\usepackage[utf8]{inputenc}

\usepackage[hyphens]{url} % not crucial - just used below for the URL
\usepackage{hyperref}

%
% \usepackage{mathptmx}      % use Times fonts if available on your TeX system
%
% insert here the call for the packages your document requires
%\usepackage{latexsym}
% etc.
%
% please place your own definitions here and don't use \def but
% \newcommand{}{}
%
% Insert the name of "your journal" with
% \journalname{myjournal}
%

%% load any required packages here



% tightlist command for lists without linebreak
\providecommand{\tightlist}{%
  \setlength{\itemsep}{0pt}\setlength{\parskip}{0pt}}

% From pandoc table feature
\usepackage{longtable,booktabs,array}
\usepackage{calc} % for calculating minipage widths
% Correct order of tables after \paragraph or \subparagraph
\usepackage{etoolbox}
\makeatletter
\patchcmd\longtable{\par}{\if@noskipsec\mbox{}\fi\par}{}{}
\makeatother
% Allow footnotes in longtable head/foot
\IfFileExists{footnotehyper.sty}{\usepackage{footnotehyper}}{\usepackage{footnote}}
\makesavenoteenv{longtable}


\usepackage{float}
\let\origfigure\figure
\let\endorigfigure\endfigure
\renewenvironment{figure}[1][2] {
    \expandafter\origfigure\expandafter[H]
} {
    \endorigfigure
}
\usepackage{lscape}
\newcommand{\blandscape}{\begin{landscape}}
\newcommand{\elandscape}{\end{landscape}}
\usepackage{booktabs}
\usepackage{longtable}
\usepackage{array}
\usepackage{multirow}
\usepackage{wrapfig}
\usepackage{float}
\usepackage{colortbl}
\usepackage{pdflscape}
\usepackage{tabu}
\usepackage{threeparttable}
\usepackage{threeparttablex}
\usepackage[normalem]{ulem}
\usepackage{makecell}
\usepackage{xcolor}
\begin{document}


\title{A hypothetical study using fake data \thanks{Grants or other notes about the article that should go on the front
page should be placed here. General acknowledgments should be placed at the
end of the article.} }


    \titlerunning{Short form of title (if too long for head)}

\author{  }

    \authorrunning{ Short form of author list if too long for running head }

\institute{
    }

\date{Received: date / Accepted: date}
% The correct dates will be entered by the editor


\maketitle

\begin{abstract}
\textbf{Background: } Brief background to study goes here. \newline \newline \textbf{Objectives: } Brief study objectives goes here. \newline \newline \textbf{Methods: } Brief description of methods goes here. \newline \newline \textbf{Results: } Brief summary of results goes here. \newline \newline \textbf{Conclusions: } Headline conclusions go here. \newline \newline \textbf{Data: } Statement about availability of study data and materials goes here. \newline \newline
\\
\keywords{
        entirely, fake, do, not, cite \and
    }

    \subclass{
                    MSC code 1 \and
                    MSC code 2 \and
            }

\end{abstract}


\def\spacingset#1{\renewcommand{\baselinestretch}%
{#1}\small\normalsize} \spacingset{1}


\textsuperscript{1} Awesome University, Shanghai\\
\textsuperscript{2} August Institution, London\\
\textsuperscript{3} Highly Ranked Uni, Montreal

\textsuperscript{*} Correspondence: \href{mailto:fake_email@fake_institute.com}{Alejandra R Scienceace \textless{}\href{mailto:fake_email@fake_institute.com}{\nolinkurl{fake\_email@fake\_institute.com}}\textgreater{}}

\hypertarget{intro}{%
\section{Introduction}\label{intro}}

Your text comes here. Separate text sections with \cite{Mislevy06Cog}.

\hypertarget{sec:1}{%
\section{Section title}\label{sec:1}}

Text with citations by \cite{Galyardt14mmm}.

\hypertarget{sec:2}{%
\subsection{Subsection title}\label{sec:2}}

as required. Don't forget to give each section
and subsection a unique label (see Sect. \ref{sec:1}).

\hypertarget{paragraph-headings}{%
\paragraph{Paragraph headings}\label{paragraph-headings}}

Use paragraph headings as needed.

\begin{align}
a^2+b^2=c^2
\end{align}


\bibliographystyle{spbasic}
\bibliography{../Data/bib.bib}


\end{document}
